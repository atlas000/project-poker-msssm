\documentclass[11pt]{article}
\usepackage{geometry}                
\geometry{letterpaper}                   

\usepackage{graphicx}
\usepackage{epstopdf}
\usepackage{amssymb}
\usepackage{epstopdf}
\usepackage{natbib}
\usepackage{amssymb, amsmath}
\usepackage{booktabs}
\usepackage{multicol}
\usepackage{color}           
\usepackage{hyphenat}
\usepackage{booktabs}
\usepackage{hyperref}

\begin{document}
\subsection{Variable documentation}
\subsubsection{Variables from generic model}

\begin{itemize}

\item	\textbf{Scripts:} \textcolor{red}{ultra\_main(um), main(m), game(g), headsUp/headsUp2(h), adjustCardValue(acv)} \\

\item	\textbf{playerP1, playerP2, \textcolor{red}{general}:} Vector with 4 entries representing all important variables from one player.
\begin{enumerate}
	\item risk factor for player
	\item capital from player
	\item card value from player, is a random number between 0 and 1
	\item total bet from player
	\item free variable for learning implementation
\end{enumerate}		

\item	\textbf{allData \textcolor{red}{um}:}  Matrix containing number of Wins from player 1 one for varying risk factors\\
	
\item	\textbf{r1, r2 \textcolor{red}{um}:}  iteration variables representing risk factors for players 1 and 2\\

\item	\textbf{betValue \textcolor{red}{m,h}:} represents the amount a player can bet on his win or the amount by which the pot can be increased per round per person \\

\item	\textbf{n \textcolor{red}{m}:}  iteration variable for determining how many games are being simulated, hence determining the accuracy of the monte carlo approach\\

\item	\textbf{riskfactorP1, riskFactorP2 \textcolor{red}{m}:}  variables representing the risk factors for both players\\

\item	\textbf{startCapital \textcolor{red}{m}:} determines the capital at the beginning of the game for both players. \\

\item	\textbf{winsP1 \textcolor{red}{m}:} amount of total wins by player 1, used as output to the function main.m  \\

\item	\textbf{winner \textcolor{red}{m,g}:} stores the the winner of the game simulated: if 0 winner is player 2, if 1 winner is player 1, used as output to the function game.m\\

\item	\textbf{counter \textcolor{red}{g}:} counts amount of hands played in one game \\

\item	\textbf{decide\_who\_starts \textcolor{red}{g}:} used to determine whose turn it is to start with betting, if 0 player 2 begins, if 1 player 1 begins  \\

\item	\textbf{pot \textcolor{red}{h}:}  stores the total amount of money betted by both players\\

\item	\textbf{capP1, capP2 \textcolor{red}{}:} output variables to headsUp.m function storing the capital of the corresponding player after having played the hand\\

\item	\textbf{newRandValue \textcolor{red}{acv}:} output to adjustCardValue.m function, stores the newly generated cardvalue  \\

\item	\textbf{sigma \textcolor{red}{acv}:} theoretically the standard deviation of a normally distributed random value, here used to determine the range of adjustement for the function adjustCardValue.m \\

\end{itemize}

\subsubsection{Variables from learning models}
\paragraph{Threshold model}
\begin{itemize}
\item	\textbf{Scripts:} \textcolor{red}{ultra\_main(um), main(m), game(g), headsUp/headsUp2(h), adjustCardValue(acv), adjustRiskFactor(arf)} \\

\item	\textbf{totalCounter \textcolor{red}{m}:} used to store total amount of hands played per game  \\

\item	\textbf{playerP1(5) \textcolor{red}{g}:} stores the risk factor of player 2 as it is currently estimated by player 1  \\

\item	\textbf{startRiskFactor \textcolor{red}{h}:} input to headsup.m function, stores the risk factor from player 2 as estimated by player 1 before the respective hand  \\

\item	\textbf{estRiskFactor \textcolor{red}{h}:} output from headsup.m function, stores the risk factor from player 2 as estimated by player 1 after the respective hand   \\

\item	\textbf{newRiskFactor \textcolor{red}{arf}:}  output from adjustRiskFactor.m function, stores newly generated risk factor\\

\item	\textbf{opponentRiskFactor \textcolor{red}{arf}:} input to function adjustRiskFactor.m, currently estimated risk factor from opponent player \\

\item	\textbf{refSurf \textcolor{red}{arf}:} two-variable function which represents amount of wins for player 1 in dependence of the respective risk factors \\

\item	\textbf{funVector \textcolor{red}{arf}:} parameterisation of refSurf at point of opponentRiskFactor \\

\end{itemize}
\end{document}
